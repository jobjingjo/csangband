\section{Older Version Information --- version.txt}

\paragraph{}This file was last updated for Angband 3.1.2.

\paragraph{}Make sure to read the newsgroup ("rec.games.roguelike.angband"), and to visit
the Official Angband Home Page ("\url{http://rephial.org}") for the most
up to date information about Angband.

\paragraph{}Angband 3.1.2 has an incredibly complex history, and is the result of a
lot of work by a lot of people, all of whom have contributed their time
and energy for free, being rewarded only by the pleasure of keeping alive
one of the best freeware games available anywhere.

\paragraph{}The version control files, if they existed, would span more than ten years
time, and more than six different primary developers.  Without such files,
we must rely on simpler methods, such as change logs, source file diffs, and
word of mouth.  Some of this information is summarised in this file.

\paragraph{}Please be sure to read the copyright information contained in the file
copying.txt.

\subsection{Brief Version History}

First came "VMS Moria", by Robert Alan Koeneke (1985).

Then came "Umoria" (Unix Moria), by James E. Wilson (1989).

\paragraph{}Details about the history of the various flavors of "Moria", the direct
ancestor to Angband, can be found elsewhere, and a note from Robert Alan
Koeneke is included in this file.  Note that "Moria" has been ported to
a variety of platforms, and has its own newsgroup, and its own fans.

\paragraph{}In 1990, Alex Cutler and Andy Astrand, with the help of other students
at the University of Warwick, created Angband 1.0, based on the existing
code for Umoria 5.2.1.  They wanted to expand the game, keeping or even
strengthening the grounding in Tolkien lore, while adding more monsters
and items, including unique monsters and artifact items, plus activation,
pseudo-sensing, level feelings, and special dungeon rooms.

\paragraph{}Over time, Sean Marsh, Geoff Hill, Charles Teague, and others, worked on
the source, releasing a copy known as "Angband 2.4.frog\_knows" at some
point, which ran only on Unix systems, but which was ported by various
people to various other systems.  One of the most significant ports was
the "PC Angband 1.4" port, for old DOS machines, which added color and
various other significant changes, only some of which ever made it back
into the official source.

\paragraph{}Then Charles Swiger (cs4w+\@andrew.cmu.edu) took over, sometime in late
1993, cleaning up the code, fixing a lot of bugs, and bringing together
various patches from various people, resulting in several versions of
Angband, starting with Angband 2.5.1 (?), and leading up to the release
of Angband 2.6.1 (and Angband 2.6.2) in late 1994.  Some of the changes
during this period were based on suggestions from the "net", and from
various related games, including "UMoria 5.5", "PC Angband 1.4", and
"FAngband".

\paragraph{}Angband 2.6.1 was primarily targetted towards Unix/NeXT machines, and
it required the use of the low level "curses" commands for all screen
manipulation and keypress interaction.  Each release had to be ported
from scratch to any new platforms, normally by creating visual display
code that acted as a "curses" emulator.  One such port was "Macintosh
Angband 2.6.1", by Keith Randall, which added support for color, and
which formed the basis for the first release of Angband 2.7.0.

\paragraph{}During the last half of 1994, Ben Harrison had been playing with
the Angband source, primarily to investigate the possibility of making
some kind of automatic player for Angband, like the old "rogue-o-matic"
program for the game "Rogue".  The difficulty of compiling a version
for the Macintosh, and the complexity of the code, prevented this, and
so Ben began cleaning up the code in various ways for his personal use.

\paragraph{}In late 1994, Charles Swiger announced that he was starting a real job
and would no longer be able to be the Angband maintainer.  This induced
some amount of uproar in the Angband community (as represented by the
Angband newsgroup), with various people attempting to form "committees"
to take over the maintenance of Angband.  Since committees have never
given us anything but trouble (think "COBOL"), there was very little
resistance when, on the first day of 1995, Ben made his code available,
calling it "Angband 2.7.0", and by default, taking over as the new
maintainer of Angband.

\paragraph{}Angband 2.7.0 was a very clean (but very buggy) rewrite that, among other
things, allowed extremely simple porting to multiple platforms, starting
with Unix and Macintosh, and by the time most of the bugs were cleaned up,
in Angband 2.7.2, including X11, and various IBM machines.  Angband 2.7.4
was released to the "ftp.cis.ksu.edu" site, and quickly gained acceptance,
perhaps helped by the OS2 and Windows and Amiga and Linux ports.  Angband
2.7.5 and 2.7.6 added important capabilities such as macros and user pref
files, and continued to clean up the source.  Angband 2.7.8 was released
to the major ftp archives as the first "stable" version in a year or so,
with new "help files" and "spoiler files" for the "online help", plus a
variety of minor tweaks and some new features.

\paragraph{}After Angband 2.7.8 was released, Ben created a web site to keep track of
all the changes made in each version (though a few may have been missed),
and acquired the use of a new develoepement ftp server to supplement the
official "mirror" server.  This web site is now permanently located at
the Official Angband Home Page (\url{http://www.thangorodrim.net/}).
Unfortunately, the next six versions were numbered Angband 2.7.9v1 to
Angband 2.7.9v6, but really each were rather major updates.  Angband 2.8.0
and 2.8.1 were released using a more normal version scheme.  Angband 2.8.2
and 2.8.3 add a few random features, clean up some code, and provide
graphics support and such for a few more platforms.

\paragraph{}After the release of Angband 2.8.3 Ben's free time was more and more
occupied by his work.  He released a beta version of Angband 2.8.5,
introducing many new features, but couldn't give as much attention to
maintaining the game as he wanted to.  Meanwhile, an "unofficial" version
by Robert Ruehlmann, incorporating three popular patches (the "Easy Patch" 
by Tim Baker, for opening doors and disarming traps without specifying 
the direction: Greg Wooledge's "Random Artifacts" patch: and Keldon Jones's
"Optional Monster AI\index{AI} Improvement"), named "2.8.3h", was gaining popularity.

\paragraph{}So in March 2000, Robert Ruehlmann offered to take over Angband and
started to fix the remaining bugs in the Angband 2.8.5 beta.  The
resulting version was to be released as Angband 2.9.0. Further 
bugfixes and a couple of new features - including many in the realms
of user-customizability, with greater control over ego-items, player
races and classes, monsters, items and artifacts - have led to the 
current version.

\paragraph{}And with the greater amount of user-customizability that is now possible,
it was inevitable that SOMEBODY would eventually go and actually do 
something with it.  Jonathan Ellis started customizing the user-editable
text files in the "edit" directory for his own personal use - originally, 
only by fixing bugs and inconsistencies (less powerful monsters being worth
more experience per kill than more powerful ones, dragons doing a decent 
amount of damage in melee, monsters with two claws and one mouth getting 
one claw and three bite attacks, and so on).  

\paragraph{}At first, this was all that could really be done with it: adding new 
monsters and items was impossible, as the limits were fixed. And so only 
three new monsters made an appearance, each of them replacing an existing 
monster in the order: and one new artifact - "The Palantir of Westernesse". 
Gameplay balance could be tweaked somewhat, by changing the level, power and 
rarity of certain items and monsters: and some changes were made, mostly 
with the attempt to reduce the notorious "triple whammy" effect of needing 
poison, confusion\index{confusion} and nether resistance (or over 550 hps, if without nether 
resistance) all at once, straight after passing 2000', forcing excessive 
scumming before this depth or risking unavoidable instant death: and then
having nothing left to do but dive straight to 4000' and scum for speed
items, missing out on some of the most interesting depths of the dungeon.
This problem, at least, could be addressed, but actual new things were less
easy to add...

\paragraph{}That all changed with Angband 2.9.1, which for the first time moved the 
limits themselves to a separate user-editable file, and allowed more 
monsters and items to be created without removing the old ones. At the 
same time, a patch by Matthias Kurzke was incorporated which allowed the 
creation of new ego-items.  Various new powers, for the player and 
monsters, were added to the game - but no items or monsters yet had these 
powers (resist fear, poison brand, lose charisma, summon greater demons,
and so on): indeed, arguably it could be said that the game had not even
adjusted properly to Ben Harrison's fractional speed system (Angband 2.7.0)
or the addition of the other attack forms such as shards, sound, chaos, 
nexus and so on (even before Ben.) 


\paragraph{}The Official Angband Home Page ("\url{http://rephial.org/}")
serves not only as the most up to date description of Angband, but also
lists changes made between versions, and changes planned for upcoming
versions, and lists various email addresses and web sites related to
Angband.

\subsection{Some of the changes between Angband 2.6.1 and 3.0.6}

\paragraph{}It is very hard to pin down, along the way from 2.6.2 to 3.0.6, exactly what
changes were made, and exactly when they were made. Most releases involved 
so many changes from the previous release as to make "diff files" not very 
useful, since often the diff files are as long as the code itself. Most of 
the changes, with the notable exception of the creation of some of the new 
"main-xxx.c" files for the various new platforms, and a few other minor 
exceptions generally noted directly in comments in the source, were written 
by Ben or Robert, either spontaneously, or, more commonly, as the result of 
a suggestion or comment by an Angband player.

\paragraph{}The most important modification was a massive "code level cleanup" for 2.7.x, 
largely completed in 2.7.8, that made all other modifications much simpler 
and safer.  This cleanup was so massive that in many places the code is no 
longer recognizable, for example, via "diff -r", often because it was 
rewritten from scratch.

\paragraph{}The second most important modification was the design of a generic "z-term.c"
package, which allows Angband to be ported to a new machine with as few as 50
lines of code.  Angband 2.9.3 thus runs without modification on many machines,
including Macintosh, PowerMac, Unix/X11, Unix/Curses, Amiga, Windows, OS2-386,
DOS-386, and even DOS-286.

\paragraph{}It would be difficult to list all of the changes between Angband 2.6.1 and
3.0.6, because many of them were made in passing during the massive code 
level cleanup.  Many of the changes are invisible to the user, but still 
provide increased simplicity and efficiency, and decreased code size, or 
make other more visable changes possible.  For example, the new "project()" 
code that handles all bolts, beams, and balls, the new "update\_view()" code 
that simplifies line of sight computation, or the new "generate()" code that 
builds new levels in the dungeon.  Many changes have been made to increase 
efficiency, including the new "process\_monsters()" and "update\_monsters()" 
functions, and the new "objdes()" and "light\_spot()" routines.  The generic 
"z-term.c" package yielded efficient screen updates, and enabled the 
efficient use of "color". 

\paragraph{}The most visible (to ordinary players) changes that happened as a result of 
Ben Harrison's\index{Ben Harrison} maintainership were (a) a far greater degree of user-
customizability as shown by the info.txt files, (b) the "fractional" speed
system, with +10 in the new scheme equalling +1 in old money, and
(c) object stacking, the ability to have more than one object in a square:
first tried in 2.7.9, completed in 2.8.2.

\paragraph{}It should also be pointed out at this point that the far cleaner nature of 
Ben's code as compared to previous versions has given many other people the 
opportunity to base code for their own Angband variants on it. And so a 
plethora of new variants have appeared, many of them far more different from 
Angband now than Angband ever was from Moria, and yet still based on Ben's 
coding ideals for Angband.

\paragraph{}For Angband 2.9.0, the first few new visible features were a random artifact 
generator (originally developed from a variant by Greg
Wooledge\index{Greg Wooledge}), an option to 
improve monster AI\index{AI} (believed to have originally started out life in a patch 
written by Keldon Jones\index{Keldon Jones}), and a patch to allow easier handling of opening and 
closing doors and disarming traps (by Tim Baker\index{Tim Baker}). For Angband 2.9.1 has also 
come such things as the ability to increase the size of the editable textfiles
and thus the number of monsters, artifacts, items, ego-items and vaults in 
the game (many new vaults were written by Chris Weisiger, some by others, and 
the number of vaults in the game at this time was doubled), and much greater 
customizability of ego-items has become possible thanks to a patch written by 
Matthias Kurzke. It is also now possible to add new character races to the 
game, and to edit the shopkeepers with respect to their greed, tolerance of 
haggling and reactions to the character based on his race. Angband 2.9.2 adds 
support for poison branded weapons to the game.  Angband 2.9.3 made the
character class itself customizable to an extent.

\subsection{A Posting from the Original Author}

\begin{verbatim}
From: koeneke@ionet.net (Robert Alan Koeneke)
Newsgroups: rec.games.roguelike.angband,rec.games.roguelike.moria
Subject: Early history of Moria
Date: Wed, 21 Feb 1996 04:20:51 GMT

I had some email show up asking about the origin of Moria, and its
relation to Rogue.  So I thought I would just post some text on the
early days of Moria.

First of all, yes, I really am the Robert Koeneke who wrote the first
Moria.  I had a lot of mail accussing me of pulling their leg and
such.  I just recently connected to Internet (yes, I work for a
company in the dark ages where Internet is concerned) and 
was real surprised to find Moria in the news groups...  Angband was an
even bigger surprise, since I have never seen it.  I probably spoke to
its originator though...  I have given permission to lots of people
through the years to enhance, modify, or whatever as long as they
freely distributed the results.  I have always been a proponent of
sharing games, not selling them.

Anyway...

Around 1980 or 81 I was enrolled in engineering courses at the
University of Oklahoma.  The engineering lab ran on a PDP 1170 under
an early version of UNIX.  I was always good at computers, so it was
natural for me to get to know the system administrators.  They invited
me one night to stay and play some games, an early startrek game, The
Colossal Cave Adventure (later just 'Adventure'), and late one night,
a new dungeon game called 'Rogue'.

So yes, I was exposed to Rogue before Moria was even a gleam in my
eye.  In fact, Rogue was directly responsible for millions of hours of
play time wasted on Moria and its descendents...

Soon after playing Rogue (and man, was I HOOKED), I got a job in a
different department as a student assistant in computers.  I worked on
one of the early VAX 11/780's running VMS, and no games were available
for it at that time.  The engineering lab got a real geek of an
administrator who thought the only purpose of a computer was WORK!
Imagine...  Soooo, no more games, and no more rogue!

This was intolerable!  So I decided to write my own rogue game, Moria
Beta 1.0.  I had three languages available on my VMS system.  Fortran
IV, PASCAL V1.?, and BASIC.  Since most of the game was string
manipulation, I wrote the first attempt at Moria in VMS BASIC, and it
looked a LOT like Rogue, at least what I could remember of it.  Then I
began getting ideas of how to improve it, how it should work
differently, and I pretty much didn't touch it for about a year.

Around 1983, two things happened that caused Moria to be born in its
recognizable form.  I was engaged to be married, and the only cure for
THAT is to work so hard you can't think about it; and I was enrolled
for fall to take an operating systems class in PASCAL.

So, I investigated the new version of VMS PASCAL and found out it had
a new feature.  Variable length strings!  Wow...

That summer I finished Moria 1.0 in VMS PASCAL.  I learned more about
data structures, optimization, and just plain programming that summer
then in all of my years in school.  I soon drew a crowd of devoted
Moria players...  All at OU.

I asked Jimmey Todd, a good friend of mine, to write a better
character generator for the game, and so the skills and history were
born.  Jimmey helped out on many of the functions in the game as well.
This would have been about Moria 2.0

In the following two years, I listened a lot to my players and kept
making enhancements to the game to fix problems, to challenge them,
and to keep them going.  If anyone managed to win, I immediately found
out how, and 'enhanced' the game to make it harder.  I once vowed it
was 'unbeatable', and a week later a friend of mine beat it!  His
character, 'Iggy', was placed into the game as 'The Evil Iggy', and
immortalized...  And of course, I went in and plugged up the trick he
used to win...

Around 1985 I started sending out source to other universities.  Just
before a OU / Texas football clash, I was asked to send a copy to the
Univeristy of Texas...  I couldn't resist...  I modified it so that
the begger on the town level was 'An OU football fan' and they moved
at maximum rate.  They also multiplied at maximum rate...  So the
first step you took and woke one up, it crossed the floor increasing
to hundreds of them and pounded you into oblivion...  I soon received
a call and provided instructions on how to 'de-enhance' the game!

Around 1986 - 87 I released Moria 4.7, my last official release.  I
was working on a Moria 5.0 when I left OU to go to work for American
Airlines (and yes, I still work there).  Moria 5.0 was a complete
rewrite, and contained many neat enhancements, features, you name it.
It had water, streams, lakes, pools, with water monsters.  It had
'mysterious orbs' which could be carried like torches for light but
also gave off magical aura's (like protection from fire, or aggrivate
monster...).  It had new weapons and treasures...  I left it with the
student assistants at OU to be finished, but I guess it soon died on
the vine.  As far as I know, that source was lost...

I gave permission to anyone who asked to work on the game.  Several
people asked if they could convert it to 'C', and I said fine as long
as a complete credit history was maintained, and that it could NEVER
be sold, only given.  So I guess one or more of them succeeded in
their efforts to rewrite it in 'C'.

I have since received thousands of letters from all over the world
from players telling about their exploits, and from administrators
cursing the day I was born...  I received mail from behind the iron
curtain (while it was still standing) talking about the game on VAX's
(which supposedly couldn't be there due to export laws).  I used to
have a map with pins for every letter I received, but I gave up on
that!

I am very happy to learn my creation keeps on going...  I plan to
download it and Angband and play them...  Maybe something has been
added that will surprise me!  That would be nice...  I never got to
play Moria and be surprised...

Robert Alan Koeneke
koeneke@ionet.net
\end{verbatim} \index{Robert Alan Koeneke} \index{Jimmey Todd}

\subsection{Previous Versions (outdated) - needs work}
\subsubsection{VMS Moria Version 4.8}
Version 0.1  : 03/25/83\\
Version 1.0  : 05/01/84\\
Version 2.0  : 07/10/84\\
Version 3.0  : 11/20/84\\
Version 4.0  : 01/20/85

\paragraph{}
Modules:

\paragraph{}
\begin{tabular}{rll}
     V1.0 & Dungeon Generator      & RAK\\
          & Character Generator    & RAK \& JWT\\
          & Moria Module           & RAK\\
          & Miscellaneous          & RAK \& JWT\\
     V2.0 & Town Level \& Misc     & RAK\\
     V3.0 & Internal Help \& Misc  & RAK\\
     V4.0 & Source Release Version & RAK\\
\end{tabular}

\paragraph{}
RAK: Robert Alan Koeneke\index{Robert Alan Koeneke}, Student/University of
Oklahoma \\
JST: Jimmey Wayne Todd Jr.\index{Jimmey Todd}, Student/University of Oklahoma

\subsubsection{Umoria Version 5.2 (formerly UNIX Moria)}
Version 4.83 :  5/14/87\\
Version 4.85 : 10/26/87\\
Version 4.87 :  5/27/88\\
Version 5.0  :  11/2/89\\
Version 5.2  :   5/9/90

\paragraph{}
James E. Wilson, U.C. Berkeley\index{James Wilson},
wilson@ernie.Berkeley.EDU, ...!ucbvax!ucbernie!wilson

\paragraph{}
Other contributors:
\paragraph{}
\begin{tabular}{ll}
D. G. Kneller         & MSDOS Moria port \\
Christopher J. Stuart & recall, options, inventory, and running code \\
Curtis McCauley       & Macintosh Moria port \\ 
Stephen A. Jacobs     & Atari ST Moria port \\
William Setzer        & object naming code \\
David J. Grabiner     & numerous bug reports, and consistency checking \\
Dan Bernstein         & UNIX hangup signal fix, many bug fixes \\
and many others...&
\end{tabular}

Copyright \copyright 1989 James E. Wilson, Robert A. Keoneke
  This software may be copied and distributed for educational, research, and
  not for profit purposes provided that this copyright and statement are
  included in all such copies.

\subsubsection{Umoria Version 5.2, patch level 1}

\subsubsection{Major Angband versions}
\paragraph{}
\begin{tabular}{ll}
Angband Version 2.0   & Alex Cutler, Andy Astrand, Sean Marsh, Geoff Hill, 
                        Charles Teague.\\
Angband Version 2.4   & 05/09/1993\\
Angband Version 2.5   & 12/05/1993 Charles Swiger \index{Charles Swiger}\\
Angband Version 2.6   & 09/04/1994 Charles Swiger\\
Angband Version 2.7   & 01/01/1995 Ben Harrison \index{Ben Harrison}\\
Angband Version 2.8   & 01/01/1997 Ben Harrison\\
Angband Version 2.9   & 10th April 2000 Robert Ruehlmann \index{Robert
Ruehlmann}
\end{tabular}

\subsection{Contributors (incomplete)}

Peter Berger, "Prfnoff", Arcum Dagsson, Ed Cogburn, Matthias Kurzke,
Ben Harrison, Steven Fuerst, Julian Lighton, Andrew Hill, Werner Baer,
Tom Morton, "Cyric the Mad", Chris Kern, Tim Baker, Jurriaan Kalkman,
Alexander Wilkins, Mauro Scarpa, John I'anson-Holton, "facade",
Dennis van Es, Kenneth A. Strom, Wei-Hwa Huang, Nikodemus, Timo Pietil\"{a},
Greg Wooledge, Keldon Jones, Shayne Steele, Dr. Andrew White, Musus Umbra,
Jonathan Ellis

