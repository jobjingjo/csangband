\section{Using the Online Help --- general.txt}
\subsection{Using the Online Help}
\paragraph{}You can press Escape (ESC) at any time to leave the online help.

\paragraph{}You can press Space to advance one page, Plus (+) to advance one half
page, or Return to advance one line. If you reach the end, you will
jump back to the start. You can press Minus (--) to back up one half
page, Underscore (\_) to back up one page, or Equal (=) to back up one
line. If you reach the start, you will stay there. Advancing by one
page is the most efficient method.

\paragraph{}You can press Hash (\#) [or Percent (\%)] to go to a specific line [or file].

\paragraph{}You can press Slash (/) [or Ampersand (\&)] to search for [or highlight]
a string. This is case-insensitive by default, but Pling (!) toggles it
to case-sensitive. Use ``\#'', ``0'', Return, ``/'', Return to restart a
search from the top of the file.

\paragraph{}Please press Space to view the rest of this file\ldots

\subsection{General Information}
\paragraph{}Angband is basically a complex single player dungeon simulation. A
player creates a character, choosing from a variety of races and
classes, and then plays that character over a period of days, weeks,
even months.

\paragraph{}The player will begin his adventure on the town level where he may
acquire supplies, weapons, armor, and magical devices by buying from
various shop owners. Then the player can descend into the Pits of
Angband, where he will explore the many levels of the dungeon, gaining
experience by killing fierce creatures, collecting powerful objects and
valuable treasure, and returning to town occasionally to buy and sell
supplies. Eventually, as the player grows more experienced, he may
attempt to win the game by defeating Morgoth, the Lord of Darkness, who
resides far below the surface.

\paragraph{}Note that Angband is a very complex game, and it may be difficult to
grasp everything at first, especially if you have never played a
``roguelike'' game before. You should probably browse through all of the
``online help files'', especially this one, before beginning any serious
adventuring\ldots

\subsection{About the game}
\paragraph{}Angband has been maintained by a succession of volunteers since it was
written in the early 1990s. The current maintainer is Andi Sidwell.

\paragraph{}Angband will run on a wide variety of systems (including Unix,
Macintosh, Windows). It is written in C, and the source code is freely
available. Creating a version for a new platform involves writing as few
as 100 lines of code and recompiling.

\paragraph{}See the Official Angband Home Page at
\url{http://rephial.org/} for up to date information about the latest
version of Angband, including a complete list of recent modifications and
a wiki about the game which includes a full user guide.

\paragraph{}You can post compliments, complaints, suggestions, bug reports, and
patches at \url{http://angband.oook.cz/}, or to the newsgroup
'rec.games.roguelike.angband'. You can also post interesting experiences
and ask for help.

\paragraph{}This version of Angband is under the GNU General Public Licence (GPL)
version 2. A copy of this licence is included with the game, in the file
copying.txt.

\paragraph{}If you're interested in the development of the game, the public bug
tracker and development road map are at \url{http://trac.rephial.org/}.

\paragraph{}The basic help files supplied with this game are more or less up to
date, but when in doubt, you should ask the newsgroup or the forum for
confirmation.

\paragraph{}Note that spoiler files are not distributed with the source since they
may spoil the game for new players (hence their name). If you want to
use them, you can obtain them from various places as with the source and
executables. Spoiler files may be placed into the ``lib/info'' directory,
or into a user specified external directory, to allow access via the
``online help'' system.

\paragraph{}Remember to tell all your friends about how much you like
Angband\ldots

\subsection{A quick demonstration}
\paragraph{}Angband is, as mentioned above, a very complex game, so you
may want to try the following quick demonstration. The following
instructions are for demonstration purposes only, and so they are
intentionally boring.

\paragraph{}For this demo, we will assume that you have never played
Angband before, that you have not requested any special ``sub-windows'',
that you have not requested any special ``graphics'' modes, that you have
a ``numeric keypad'' on your computer, and that you are using the default
options, including, in particular, the ``original'' command set. If any of
these assumptions are incorrect, you will need to keep in mind that this
demo may not work. There are many ways to view this file while playing, in
particular, you should be able to view it using the ``online help'' built
into the game.

\paragraph{}Any time you see the ``--more--'' prompt, read the message and
press space.  This takes precedence over any other instructions. At any
other prompt, for example, if you accidentally hit a key, you can normally
``cancel'' the action in progress by pressing escape.

\paragraph{}When the game starts up, depending on what platform you are
using, you may be taken directly to the character creation screen, or you
may have to ask to create a new character by using the File menu. In
either case, you will be shown the character information screen, and you
will be given a series of choices. For this demo, press ``a'' four times
to select a ``female human warrior'' character with the point-based stat
allocation system. You will now be presented with a description of your
character. Look over the description briefly, there is a lot of
information here, and most of it will not make any sense. Press enter
three times and your character will be placed into the ``town''.

\paragraph{}You should now be looking at the basic dungeon interaction screen. To
the left is some information about your character. To the right is an
overhead view of the town. Nothing happens in Angband while the game is
waiting for you to specify a command, so take a good look at the town.
You will see a variety of symbols on the screen. Each symbol normally
represents a terrain feature, an object, or a monster. The ``@'' symbol
is special, it represents your character. You can use the ``/'' command
to find out what a given symbol represents. Press ``/'' then ``@'' now to
verify the meaning of the ``@'' symbol.

\paragraph{}The solid blocks (which may be ``\#'' symbols on some systems) around the
edge of the town represent the walls that surround the the town. You
cannot leave the town above ground, although some games derived from
Angband (called ``variants'') have an overground element.

\paragraph{}The large rectangles represent stores. The ``numeric'' symbols represent
an ``entrance'' to a store. The ``.'' symbols represent the ``floor''. It is
currently daytime, so most of the town should consist of stores and
illuminated floor grids.

\paragraph{}Any ``alphabetic'' symbols always represent monsters, where the word
``monsters'' specifies a wide variety of entities, including people,
animals, plants, etc. Only a few ``races'' of monsters normally appear in
town, and most of them are harmless (avoid any mercenaries or veterans
if you see them). The most common ``monsters'' in town are small animals
(cats and dogs) and townspeople (merchants, mercenaries, miscreants,
etc).

\paragraph{}Now use the ``l'' command to ``look'' around. This will cause the cursor to
be moved onto each ``interesting'' square, one at a time, giving you a
description of that square. The cursor always starts on the square
containing your character. In this case, you will see a message telling
you that your character is standing on a staircase. Keep pressing space
until the prompt goes away.

\paragraph{}Now press ``i'', to display your character's ``inventory''. New characters
start out with some objects to help them survive (though there is an
option to start with more money instead). Your character will have some
food, a potion, some torches, and a scroll. Press ``e'' to see what you
are wearing. You will find you are wearing armour on your body,
wielding a broad sword and lighting the way with a torch. You have many
other equipment slots but they are all currently empty.

\paragraph{}Press ``t'' to take something off. Note that the equipment listing is
reduced to those objects which can actually be taken off. Press ``g'' to
take off the armour, and then press ``e'' again. Note that the armour is
no longer shown in the equipment. Press escape. Press ``w'' to wield
something and observe that the inventory listing is reduced to those
objects which can actually be wielded or worn, press ``e'' to put the
armour back on.

\paragraph{}Monsters can only move after you use a command which takes
``energy'' from
your character. So far, you have used the ``w'' and ``t'' commands, which
take energy, and the ``e'', ``i'', ``l'', and ``/'' commands, which are
``free''
commands, and so do not take any energy. In general, the only commands
which take energy are the ones which require your character to perform
some action in the world of the game, such as moving around, attacking
monsters, and interacting with objects.

\paragraph{}If there were any monsters near your character while you were
experimenting with the ``w'' and ``t'' commands, you may have seen them
``move'' or even ``attack'' your character. Although unlikely, it is even
possible that your character has already been killed. This is the only
way to lose the game. So if you have already lost, simply exit the game
and restart this demo.

\paragraph{}One of the most important things that your character can do is move
around. Use the numeric keys on the keypad to make your character move
around. The ``4'', ``6'', ``8'', and ``2'' keys move your character west, east,
north, and south, and the ``7'', ``9'', ``1'' and ``3'' keys move your character
diagonally. When your character first moves, observe the ``$>$'' symbol
that is left behind. This is the ``staircase'' that she was standing on
earlier in the demo - it is the entrance to the dungeon.

\paragraph{}Attempting to stay away from monsters, try and move your character
towards the entrance to the ``general store'', which is represented as a
``1'' on the screen. As your character moves around, use the ``l'' command
to look around. You can press escape at any time to cancel the looking.
If you die, start over.

\paragraph{}One of the hardest things for people to get used to, when playing games
of this nature for the first time, is that the character is not the same
as the player. The player presses keys, and looks at a computer screen,
while the character performs complex actions, and interacts with a
virtual world. The player decides what the character should do, and
tells her to do it, and the character then performs the actions. These
actions may induce some changes in the virtual world. Some of these
changes may be apparent to the character, and information about the
changes is then made available to the player by a variety of methods,
including messages, character state changes, or visual changes to the
screen. Some changes may only be apparent to the player.

\paragraph{}There are also a whole set of things that the player can do that can not
even be described in the virtual world inhabited by the character, such
as resize windows, read online help files, modify colormaps, or change
options. Some of these things may even affect the character in abstract
ways, for example, the player can request that from now on all monsters
know exactly where the character is at all times. Likewise, there are
some things that the character does on a regular basis that the player
may not even consider, such as digesting food, or searching for traps
while walking down a hallway.

\paragraph{}To make matters worse, as you get used to the difference between the
player and the character, it becomes so ``obvious'' that you start to
ignore it. At that point, you find yourself merging the player and the
character in your mind, and you find yourself saying things like ``So
yesterday, I was at my friend's house, and I stayed up late playing
Angband, and I was attacked by some wild dogs, and I got killed by a
demon, but I made it to the high score list'', in which the pronoun
changes back and forth from the real world to the virtual one several
times in the same sentence. So, from this point on you may have to
separate the player and the character for yourself.

\paragraph{}So anyway, keep walking towards the entrance to the general store until
you actually walk into it. At this point, the screen should change to
the store interaction screen. You will see the name of the shop-keeper,
and the name of the shop, and a list of objects which are available. If
there are more than twelve different objects, you can use the space or
arrow keys to scroll the list of objects. The general store is the only
store with a fixed inventory, although the amount of various items may
vary. One of the items sold here are flasks of oil. Press ``down'' to
highlight the line with flasks of oil and press the ``p'' key to purchase
some. If you are asked how many you want, just hit enter. Any time you
are asked a question and there is already something under the cursor,
pressing return will accept that choice. Hit enter to accept the price.
Many commands work inside the store, for example, use the ``i'' command to
see your inventory, with the new flask of oil. Note that your inventory
is always kept sorted in a semi-logical order, so the indexes of some of
the objects may change as your inventory changes.

\paragraph{}Purchase a few more flasks of oil, if possible: this time, when asked
how many you want, press ``3'' then return to buy three flasks at once.
Flasks of oil are very important for low level characters, because not
only can they be used to fuel a lantern (when you find one), but also
they can be ignited and thrown at monsters from a distance. So it is
often a good idea to have a few extra flasks of oil. Press escape to
leave the store. If you want, take time to visit the rest of the
stores. One of the buildings, marked with an ``8'', is your ``home'', and
is not a real store. You can drop things off at home and they will stay
there until you return to pick them up. The interface is exactly the
same as a store, but there is no payment.

\paragraph{}Now move to the staircase, represented by the ``$>$'' symbol,
and press ``$>$'', to go down the stairs. At this point, you are in the
dungeon. Use the ``l'' command to look around. Note that you are standing
on a staircase leading back to town. Use the ``$<$'' command to take the
stairs back to town. You may find that any townspeople that were here
before have disappeared and new ones have appeared instead. Now use the
``$>$'' command to go back down the stairs into the dungeon. You are now
in a different part of the dungeon than you were in before. The dungeon
is so huge, once you leave one part of the dungeon, you will never find
it again.

\paragraph{}Now look the screen. Your character may be in a lit room, represented
as a large rectangle of illuminated floor grids (``.''), surrounded by
walls. If you are not in a lit room, keep going back up to the town and
back down into the dungeon until you are. Now look around. You may see
some closed doors (``+'') or some open doors (`` ' '') or some open exits
(``.'') in the walls which surround the room. If you do not, keep playing
the stairway game until you are in such a room. This will keep the demo
simple.

\paragraph{}Now look around using the ``l'' command. You may see some monsters and/or
some objects in the room with you. You may see some stairs up (``$<$'') or
some stairs down (``$>$''). If you see any monsters, move up next to the
monster, using the movement keys, and then try and move into the
monster. This will cause you to attack the monster. Keep moving into
the monster until you kill the monster, or it runs away, or you die. If
you die, start a new game. If the monster runs away, ignore it, or
chase it, but do not leave the room. Once all the monsters in the room
are dead or gone, walk on top of any objects in the room. Press ``g'' to
get the object, and it will be added to your inventory. If there are
any closed doors (``+'') in the room walk up next to them and press
``o'' and then the direction key which would move you into the door, which
should attempt to ``open'' the door.

\paragraph{}Now use the movement keys to explore the dungeon. As you leave the
room, you will probably notice that your character cannot see nearly as
far as she could in the room. Also, you will notice that as she moves
around, the screen keeps displaying some of the grids that your
character has seen. Think of this as a kind of ``map'' superimposed on
the world itself, the player can see the entire map, but the character
can only see those parts of the world which are actually nearby. If the
character gets near the edge of the ``map'' portion of the screen the
entire map will scroll to show a new portion of the world. Only about
ten percent of the dungeon level can be seen by the player at one time,
but you can use the ``L'' command to look at other pieces of the map. Use
the ``.'' key, then a direction, to ``run'' through the dungeon. Use the
``R'' key, then return, to force your character to ``rest'' until she has
recovered from any damage she incurs while attacking monsters. Use the
``M'' key to see the entire dungeon level at once, and hit escape when
done. If your food rations are still at index ``a'' in your inventory,
press ``E'', ``a'' to eat some food. If your oil is still at index ``b'' in
your inventory, and there is a monster nearby, press ``v'', ``b'', ``{'}'' to
throw a flask of oil at the nearest monster. To drop an item from your
inventory, press ``d'' plus the index of that item. You can use the
``\^{}X'' key to quit and save the game.

\paragraph{}You now know enough to play a quick game of Angband. There is a lot
more for you to learn, including how to interpret information about your
character, how to create different kinds of characters, how to determine
which equipment to wield/wear, how to use various kinds of objects, and
how to use the more than fifty different commands available to your
character. The best resource for learning these things is the online
help, which include, among other things, a complete list of all commands
available to you, and a list of all the symbols which you may encounter
in the dungeon, and information about creating new characters.
